% !TEX root = ../thesis.tex

A piece of solid material, small enough to fit in the palm of a hand, contains of the order of $ 10^{23} $ particles. It is obvious that these particles interact with one another, if not they would go everywhere and the solid disintegrate. The theoretical treatment of interactions between such a humongous number of particles belongs to the field of many-body physics. In order to accurately model the interplay between such an amount of particles, a quantum mechanical framework referred to as quantum field theory is used. The collective behavior of particles is then described, in a quantum mechanical setting, as quasiparticles. Like the name suggests, a quasiparticles is not a fundamental particle, but rather an emergent phenomena with similarities to a particle. An example would be the collective excitation associated with vibrations of atoms in a lattice structure which is nothing but a quantum of a sound wave, known as a phonon.

There exist several other quasiparticles besides the phonon, one of which is the polaron \cite{AdvancesInPolaronPhysics}. This quasiparticle describes the dynamic which comes about when an electron interacts with the polarization in its surroundings. One particular scenario is that of a low energetic electron brought into a continuous modeled ionic crystal. This situation is described by the Fr�hlich Hamiltonian \cite{electronsInLatticeFields, RichardFeynman, Iadonisi} and the resulting polaron is appropriately referred to as the (large) Fr�hlich polaron. When propagating through the crystal, the introduced electron will still behave as free, but with an enhanced effective mass due to being accompanied by the surrounding polarization. Since the polarization arise due to deformations of the lattice, which in terms may be described by phonons, one say that the electron is dressed in a cloud of phonons. This collective behavior of the electron together with the phonon cloud is what constitutes the Fr�hlich polaron.

For a long time properties of quasiparticles has been calculated analytically using perturbation theory. This is an extremely time-consuming machinery which is not guaranteed to work in all circumstances. Fortunately however, there today exist computer intensive methods for calculating properties of the quasiparticles. One such method is the diagrammatic Monte Carlo (DMC) method developed by Prokof'ev et al. \cite{MishchenkoA.2000DqMC}  which has proved to be successful \cite{prokof2008fermi, kozik2010diagrammatic, burovski2001diagrammatic, kulagin2013bold}. This is a Metropolis-Hastings based Monte Carlo method which performs a random walk between Feynman diagrams of a perturbations series in order to simulate the quantity this series represent.

In particular the the DMC method could be implemented to calculate the electronic single-particle Green's function of the Fr�hlich Hamiltonian from which quantities of the polaron then might be extracted. Perhaps not so surprisingly to the reader, this is exactly what is investigated in this thesis. Three different variants, two bare schemes and one bold-line scheme, are implemented in a self made c++ program and the result obtained is presented in this thesis. Without further ado, let's jump into all the technicalities!