% !TEX root = ../thesis.tex

\section*{Comments guide}

\begin{itemize}
	\item \image{Something about an image}

	\item \motivation{Motivation for myself}

	\item \question{Question about something}

	\item \flaw{There is something wrong, a flaw}

	\item \todo{Todo, things which need to be done at some point}
\end{itemize}

%\newpage
%
%\section*{Notations}
%
%\subsection*{Units}
%
%Throughout this text $ k_\text{B} = \hbar = 1 $ has been used. With this notation there no difference between wave vectors and momentum.
%
%
%\subsection*{Fourier transform \todo{remove this since it is not the case}}
%
%A function $ f(x) $ which is defined and integrable on the interval $ x \in [-L/2, \, L/2] $ may be represented as a Fourier series. In this thesis the following convention will be used
%
%\begin{equation}
%	c_k
%	\equiv \int_{-L/2}^{L/2} f(x) \, e^{-i k \cdot x} \diff x
%	\quad \Rightarrow \quad
%	f(x)
%	= \frac{1}{L} \sum_{k} c_k \, e^{i k \cdot x} \,.
%\end{equation}
%where the wave vectors are given by $ k = 2 \pi n / L $ and $ n \in \mathbb{Z} $. In the limit $ L \rightarrow \infty $ the Fourier transform is defined accordingly
%
%\begin{equation}
%	\hat f(k)
%	\equiv \int_\mathbb{R} f(x) \,e^{-i k \cdot x} \diff x
%	\quad \Rightarrow \quad
%	f(x)
%	= \frac{1}{2\pi} \int_\mathbb{R} \hat f(k) \, e^{i k \cdot x} \diff k \,.
%\end{equation}
%
%\question{How does this definition affect the conjugate variables in the derivation of the Fr�hlich Hamiltonian?}
%
%\todo{
%Perhaps mention something about DOS and motivate
%
%\begin{equation}
%	\frac{1}{V} \sum_{\vec k} \cdots \rightarrow \int \frac{\diff^3k}{(2\pi)^3} \cdots \,.
%\end{equation}
%
%% http://web.ift.uib.no/AMOS/nazila/LaserAndLight/node8.html
%}


\todo{
\section*{TODO's}

\begin{itemize}
	\item Sometime: momentum $ \rightarrow $ wave vector
	
	\item Remove Feynman diagram helping backgrounds
	
	\item Diagram sum $ \rightarrow $ Diagram series
\end{itemize}
}
