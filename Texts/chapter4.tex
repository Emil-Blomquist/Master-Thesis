% !TEX root = ../thesis.tex

The main focus of this thesis has been to investigated three types DMC schemes for numerically simulating the electronic single-particle Green's function of the Fr�hlich Hamiltonian, also known as the Fr�hlich polaron. To achieve this a program has been written in c++ which has used the Message Passing Interface in order to parallelize the code over several cores. The numerical results presented in this thesis are entirely obtained using this program.

The difference in results between the two bare schemes, one calculating $ \Gt $ and the other one the proper self-energy $ \Sigma^* $ of the former, is presented in figure \ref{fig:dysonDifference}. From this figure it is evident that the difference is decreasing with a finer discretization, i.e.\ a smaller $ \Delta \tau $, as expected. One should also mention that the high frequency noise present in the data in this figure originates from the $ \Gt $'s which have not been obtained using Dyson equation. As mentioned in section \ref{sec:DysonImplementation}, any such noise present in $ \Sigma^*(\vec p, \tau) $ will be greatly suppressed when undergoing Dyson equation.

The bold-line scheme, which is making use of Dyson equation self-consistently did not perform as well as the bare schemes. The main reason for this being the extremely poor statistics when starting to converge with the true Green's function, as is evident from figures \ref{fig:boldNmax8} and \ref{fig:boldNmax9}. Once again the high frequency noise is seen to have been reduced significantly when comparing $ \Gt_k $ to $ S_k $ due to Dyson equation. However, the noise present in $ S_k $ was so substantial that it no longer merely made an impact to the high frequency parts of the spectrum, but lower frequency parts were also largely affected. As low frequency noise is not suppressed as much as noise of higher frequencies when performing Dyson equation, this low frequency noise made an small but noticeable imprint on the Green's function once back in the $ \tau $-space. This is what caused the relatively large uncertainty for the quantities corresponding to $ n_\text{max} = 8, 9 $ in figures \ref{fig:EandZvsN} and \ref{fig:EandZvsK}.

Without somehow being able to improve the statistics, it would not be meaningful to continue the bold-line scheme for larger $ n_\text{max} $ since the computational cost in order to reduce the noise would be enormous (a total of 812 core days was already spent for the simulation using $ n_\text{max} = 9 $). As mentioned in section \ref{sec:boldResults}, the poor statistics arose from the fact that very little time was spent at the zeroth order diagram. Therefore the fluctuations present in the histogram $ N $ could not be reduced when divided by $ N_0 $ in (\ref{eq:skeletonApproximation}). To partly resolve this one could probably have redefined $ \alpha \rightarrow \alpha' = \alpha \, f(n) $ where $ f(n) \leq 1 $ would be a function of the diagram order $ n $ so that it would be less aspiring for the Markov process to sample higher order diagrams. Knowing the expression for $ f(n) $ one could then have extrapolated the obtained result to the case of $ f(n) = 1 $ and in this way acquire result with less noise.

Even without the perfect data it would be surprising, considering figures \ref{fig:EandZvsN} and \ref{fig:EandZvsK}, if $ E_0 $ and $ Z_0 $ of the bold-line scheme would not converge to the true values as $ n_\text{max} $ and the bold-line iteration $ k $ were increased. If improvements to the code was made and this achieved, it would have been interesting to look at the effects of the discretization also for this scheme.

Pleased with the overall agreement between the different DMC schemes, both the energy $ E_0 $ and the $ Z_0 $-factor of the polaron was computed as functions of external momentum $ \vec p $ and interaction parameter $ \alpha $. Since the bare scheme for $ \Sigma^* $ was deemed as best performing, this was the one used. The result turned out to be in excellent agreement with previous obtained data \cite{PhysRevLett.81.2514, MishchenkoA.2000DqMC}.

During this thesis the dispersion relation of the phonons has been assumed to be dispersionless. But how would a different dispersion relation, one which is not dispersionless, affect the quantities $ E_0 $ and $ Z_0 $? This would have been interesting to investigate, if more time would have been available.