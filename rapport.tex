% !TEX encoding = UTF-8 Unicode
% !TEX TS–program = pdflatexmk

\documentclass[12pt]{report}

\usepackage{amsmath, amsfonts, amssymb, graphicx, verbatim, nicefrac}
\oddsidemargin=0in
\evensidemargin=0in
\textwidth=6.25in
\headsep=0pt
\headheight=0pt
\topmargin=0in
\textheight=9in
\setlength{\footskip}{35pt}
\setlength\parindent{0pt}                                               % no indents for paragraphs

\usepackage[utf8]{inputenc}                                             % UTF8 characters available
\usepackage[T1]{fontenc}
\usepackage[english, swedish]{babel}
\usepackage{titlesec}                                                   % for manipulating titles
\usepackage[titletoc]{appendix}                                         % Options makes nice TOC
\usepackage{bm}                                                         % for bold face vectors
\usepackage{color}
\usepackage{url}
\usepackage{varwidth}                                                   % for boxing
\usepackage{mathtools}                                                  % "dcases" in equations etc...
\usepackage[parfill]{parskip}                                           % vertical space in between paragraphs
\usepackage{bbold}                                                      % eye symbol
\usepackage[allowzeroexp=true,numdigits,obeymode=true,redefsymbols,textcelsius,textdegree,textminute,textmu,load={abbr,addn},decimalsymbol=comma]{siunitx}                                                             % SI-units
\usepackage{float}												        % figures float with text properly


\usepackage{graphicx}
\usepackage{caption}
\usepackage{subcaption}


\newenvironment{abstractpage}
  {\cleardoublepage\vspace*{\fill}\thispagestyle{empty}}
  {\vfill\cleardoublepage}
\renewenvironment{abstract}[1]
  {\bigskip\selectlanguage{#1}%
   \begin{center}\bfseries\abstractname\end{center}}
  {\par\bigskip}

\newcommand\eye{\mathbb{1}}                                             % eye symbol
\newcommand\zahlen{\mathbb{Z}}                                          % eye symbol
\renewcommand{\vec}[1]{\boldsymbol{\mathbf{#1}}}                        % bold face vectors
\renewcommand{\ss}[1]{\textsuperscript{#1}}                             % superscript

\newcommand*\diff{\mathop{}\!\mathrm{d}}
\newcommand*\Diff[1]{\mathop{}\!\mathrm{d^#1}}

\addto{\captionsenglish}{\renewcommand{\bibname}{References}}           % Rename Bibleography




\newcommand{\Gt}{\mathcal{G}}


\definecolor{image}{rgb}{1, 0, 1}
\definecolor{motivation}{rgb}{0, 0.8, 0}
\definecolor{question}{rgb}{0.8, 0.5, 0}
\definecolor{flaw}{rgb}{1, 0, 0}
\definecolor{todo}{rgb}{0, 0, 1}

\newcommand{\image}[1]{{\leavevmode\color{image}#1}}
\newcommand{\motivation}[1]{{\leavevmode\color{motivation}#1}}
\newcommand{\question}[1]{{\leavevmode\color{question}#1}}
\newcommand{\flaw}[1]{{\leavevmode\color{flaw}#1}}
\newcommand{\todo}[1]{{\leavevmode\color{todo}#1}}




\title{
	\vspace{-2.5cm}
	\begin{center}
		\includegraphics[width=2.5cm]{{"Images/KTH logo/KTH_CMYK"}.eps}\\[-1mm]
		\hspace{-3mm} {\tiny {\sf Theoretical Physics}}
	\end{center}
	\vspace{2cm}
	\noindent\rule{16cm}{0.4pt}\\[2ex]
	Diagrammatic Monte Carlo\\
	\noindent\rule{16cm}{0.4pt}\\[3ex]
}

\author{
	Blomquist, Emil \\
	\texttt{emilbl@kth.se} \\[3em]
	Master of Science Thesis \\
	Supervisor: Egor Babaev \\[1em]
}

\date{
    \dateenglish
    \today
}

\begin{document}

\maketitle
\begin{abstractpage}
\begin{abstract}{english}
\todo{Write an abstract}
\end{abstract}
\begin{abstract}{swedish}
\todo{Skriv en sammanfattning}

\end{abstract}
\end{abstractpage}


\section*{Comments guide}

\image{Something about an image}

\motivation{Motivation for myself}

\question{Question about something}

\flaw{There is something wrong, a flaw}

\todo{Todo, things which need to be done at some point}

\newpage

\section*{Notations}

\todo{Throughout this text $ k_\text{B} = \hbar = 1 $ has been used.}

\question{Something else about notations?}


\selectlanguage{english}

\tableofcontents

\chapter{Introduction}

\todo{Here we shall summarise what this chapter is about}

\section{Many-particle physics}



\chapter{Background Material}

\todo{Here we shall summarise what this chapter is about}

\section{A few words about finite temperature formalism}

\todo{Point out that we are summarizing from Fetter Walecka.}

When working with field theories at $ T = 0 $, even if the ground state is not the vacuum one, one does not need to be concerned about thermodynamics. At $ T \neq 0 $ however, unless the ground state is the vacuum, things are different. Although the formalism at finite temperature is interesting in itself, and much time could be spent discussing it, will only mention some similarities and differences relative the $ T = 0 $ formalism. \todo{Perhaps we will not even cover that, but only the absolute basics}

Treating a system with variable number of particles at finite temperature, it is beneficial to work with the grand canonical ensemble. The partition function is then given by $ Z_\text{G} = \text{Tr} \, e^{-\beta (\hat H - \mu \hat N)} = \text{Tr} \, e^{-\beta \hat K} $. Here $ \beta $ is the inverse temperature (recall that we use units where $ k_\text{B} = 1 $), $ \hat H $ is the Hamiltonian, $ \hat N $ is the number operator, $ \mu $ is the chemical potential and the trace is taken over a complete set of states. Introduced was also the grand canonical Hamiltonian $ \hat K \equiv \hat H - \mu \hat N $. In accordance to the partition function, the statistical operator is then defined as $ \hat \rho_\text{G} = e^{-\beta \hat K} / Z_\text{G} $.

At zero temperature the Heisenberg picture was used when calculating observables. In this picture the operators are responsible for the time evolution of the observables whilst the state vectors are made time independent. Moreover, an operator in the Heisenberg picture is related to the corresponding operator in the Schrödinger picture through $ \hat O_\text{H}(t) \equiv e^{i \hat H t} \, \hat O_\text{S} \, e^{-i \hat H t} $. To calculate an observable from an operator, one would at $ T = 0 $ simply sandwiches the operator in between the ground state vector as such, $ O(t) = \langle \Psi_0 |  \hat O_\text{H}(t)  | \Psi_0 \rangle $. At finite temperature however, the observable is not given by merely an expectation value from a single state. As pointed out before, the statistics need now to be woven in. In accordance with the partition function, the observable should instead be sandwiched in between a complete set of states, and each weighted using the statistical operator,

\begin{equation}
	\label{eq:finiteTempObservable}
	O(t)
	= \sum_\nu \langle \Psi_\nu |  \hat \rho_\text{G} \, \hat O_\text{H}(t)  | \Psi_\nu \rangle
	\equiv
	\text{Tr} \{ \hat \rho_\text{G} \, \hat O_\text{H}(t) \} \,.
\end{equation}

In equation (\ref{eq:finiteTempObservable}) above, we have both real and imaginary exponents coming from the thermodynamics and quantum mechanics respectively \question{(is it correct to say that, or do I need to be more specific?)}. In order to proceed, it would be convenient if one could treat the two types of exponents on an equal footing, and in order to do so a new picture is introduced. This is a modified Heisenberg picture where a Wick rotation in time $ \tau = i t $ is preformed, and the Hamiltonian is exchanged in favor of the grand canonical Hamiltonian. An operator in this picture is then related to itself in the Schrödinger picture as $ \hat O_\text{K}(\tau) \equiv e^{\hat K \tau} \, \hat O_\text{S} \, e^{-\hat K \tau} $. From this a perturbative treatment of an expectation value can be developed in a similar manner to what is done at zero temperature.


\begin{equation}
	\text{\todo{Insert equation of an observable expressed as a perturbation series.}}
\end{equation}

\question{Should I write about having to do an analytical continuation of the temperature greens function in order to obtain frequencies and lifetimes of the excited states at finite temperature? Since we are not concerned about that in this thesis I recon that would be a waste of time.}


\subsection{Green's function}

Before introducing the concept of a Green's functions it is probably a good idea to first mention field operators. These operators are superpositions of creation and annihilation operators

\begin{equation}
	\label{eq:fieldOperator}
	\hat \psi (\vec x) \equiv \sum_{\nu} \hat c_{\nu} \psi_{\nu}(\vec x)
	\; , \quad
	\hat \psi^\dagger (\vec x) \equiv \sum_{\nu} \hat c_{\nu} \psi_{\nu}(\vec x) \,.
\end{equation}

Here $ \psi_{\nu}(\vec x) $ is a single-particle wave function of an eigenstate with eigenvalue $ \nu $ of some operator $ \hat o_\nu $ for which $ \hat c^\dagger _\nu $ and $ \hat c _\nu $ are the corresponding many-body creation and annihilation operators respectively. The sum is taken over the complete set of quantum numbers $ \{ \nu \} $. It happens to be so, that it is extremely convenient to work with field operators when converting operators from first-quantized to second-quantized form. Also the Green's functions make use of the field operators.

The Green's function, playing an important role when analyzing an interacting many-particle system, is the quantity of greatest interest in this thesis. At zero temperature, in position and time representation, the single-particle Green's function is defined as

\begin{equation}
	i G_{\alpha \beta} (x, x')
	\equiv \frac{\langle \Psi_0 | T[\hat \psi_{\text{H} \alpha} (x) \hat \psi^\dagger_{\text{H} \beta} (x')] |\Psi_0 \rangle}{\langle \Psi_0 |\Psi_0 \rangle} \,.
\end{equation}

Here $ | \Psi_0 \rangle $ is the ground state vector in the Heisenberg picture, $ \hat \psi_{\text{H} \alpha}(x) $ is the $ \alpha $-th component of a field operator also in the Heisenberg picture, $ T[ \cdots ] $ orders the operators according to their time and $ x $, $ x' $ being position-time four vectors.

\todo{Discuss the interpretation of a Green's function: $ G^2 $ is the probability of finding a particle at $ x $ which was introduced into the system at $ x' $.}

This Green's function contains observable properties such as both the ground state energy and the excitation spectrum of the system.  By using the Lehman representation, the expectation value of any single-particle operator in the ground state of the system might also be extracted \cite{quantumTheoryOfManyParticleSystems}.

Above zero temperature, now working with an imaginary-time $ \tau $, the Green's function of interest to us is the temperature Green's function. In the imaginary-time and position representation, the corresponding single-particle temperature Green's function is defined through

\begin{equation}
	\Gt_{\alpha \beta} (x, x')
	\equiv \text{Tr} \{ \hat \rho_\text{G} \, T_\tau [\hat \psi_{\text{K} \alpha}(x) \hat \psi^\dagger_{\text{K} \alpha}(x') ] \} \,.
\end{equation}

\question{What about the sign!? According to wikipedia the sign should not be there. The sign is chosen differently compared to Fetter Walecka.}

Here $ \psi_{\text{K} \alpha}(x) $ is the $ \alpha $-th component of a field operator in the modified Heisenberg picture, $ T_\tau [\cdots] $ orders the operators according to their value of $ \tau $,  $ \text{Tr} \{ \rho_\text{G} \cdots \} $ is the same weighted sum of inner products as mention before and $ x $, $ x' $ being position-imaginary-time four vectors. Since the temperature single-particle Green's function is not a function of time, it is only possible to directly extract from it observables which neither dependent on time, that is, thermodynamical properties. In order to obtain time dependent properties, one must first relate the temperature Green's function to a real-time Green's function. This is something which wont be of importance to this thesis and thus not discussed. From here on we will stick to the finite temperature formalism and whenever a Green's function is mention it will be understood that a temperature Green's function intended.

Assuming the Hamiltonian in the Schrödinger picture is time independent, and also such that it commutates with the momentum operator $ \hat{ \vec p} $, then the single-particle thermal Green's function will depend only on the difference $ \vec x - \vec x' $ and $ \tau - \tau' $. For the purpose of this thesis it is also sufficient to further assume that the spin dependency of the Green's function might be factored out as

\begin{equation}
	\Gt_{\alpha \beta}(\vec x, \tau) = \delta_{\alpha \beta} \, \Gt(\vec x, \tau) \,.
\end{equation}

Here we have transformed the parameters $ \vec x - \vec x' \rightarrow \vec x $ and $ \tau - \tau' \rightarrow \tau $ for a shorter notation. Since the spin dependency is trivial, we will drop it and from now on refer to the single-particle Green's function as $ \Gt (\vec x, \tau) $.

For reasons to become evident later, it is necessary to work within the momentum-imaginary-time representation rather than the position-imaginary-time representation. Hopefully it should not come as a surprise, that in the limit of infinite system size, the single-particle Greens function in momentum space is obtain by the Fourier transform,

\begin{equation}
	\label{eq:defGpt}
	\Gt (\vec x, \tau) = \int \frac{\diff^3 k}{(2 \pi)^3} \,  e^{i \vec k \cdot \vec x} \, \Gt (\vec k, \tau) \,.
\end{equation}

For the sake of the thesis it is not necessary to keep the ground state ensemble of the system completely general. So in order to simplify the following analysis it is permissible to assume the ground state to be empty \question{(do we also need to assume $ T = 0 $ ?)} (this is actually the ground state to be used in the upcoming diagrammatic Monte Carlo algorithm \todo{(better word needed)}). This implies that $ \text{Tr}\{ \cdots \} \rightarrow \langle \text{vac} | \cdots | \text{vac} \rangle $.


Assuming the system to be cubic, with a finite volume $ V = l^3 $, accompanied also by periodic boundary conditions, the single-particle wave functions of the momentum eigenstates are found to be

\begin{equation}
	\psi_{\vec k} (\vec x) = \frac{e^{i \vec k \cdot \vec x}}{\sqrt V} \,.
\end{equation}

The components of the allowed momenta are then

\begin{equation}
	\vec k_i = \frac{2\pi}{l} n_i
	\; ; \quad 
	n_i = 0, \, \pm 1, \, \pm 2, \cdots \,.
\end{equation}

Using this set of quantum numbers in the definition (\ref{eq:fieldOperator}) of the field operator, it is possible to rewrite

\begin{equation}
	\label{eq:Gxt2Gpt}
	\begin{split}
		\Gt (\vec x, \tau)
		&= \sum_{\vec k, \vec p} \psi_{\vec k}(\vec x) \psi^\dagger_{\vec p}(\vec 0) \langle \text{vac} | \hat \rho_\text{G} \, T_\tau [\hat c_{\vec k}(\tau) \hat c^\dagger_{\vec p}(0)] | \text{vac} \rangle \\
		&= \theta(\tau) \sum_{\vec k, \vec p} \psi_{\vec k}(\vec x) \psi^\dagger_{\vec p}(\vec 0) \delta_{\vec k, \vec p} \langle \text{vac} | \hat c_{\vec k}(\tau) \hat c^\dagger_{\vec k}(0) | \text{vac} \rangle \\
		&= \theta(\tau) \sum_{\vec k} \psi_{\vec k}(\vec x) \psi^\dagger_{\vec k}(\vec 0) \langle \text{vac} | \hat c_{\vec k}(\tau) \hat c^\dagger_{\vec k}(0) | \text{vac} \rangle \\
		&= \theta(\tau) \frac{1}{V} \sum_{\vec k} e^{i \vec k \cdot \vec x} \langle \text{vac} | \hat c_{\vec k}(\tau) \hat c^\dagger_{\vec k}(0) | \text{vac} \rangle \\
	\end{split}
\end{equation}

Taking the limit $ l \rightarrow \infty $ the separation between the eigen-momenta become infinitesimally small, and one should thus replace with

\begin{equation}
	\frac{1}{V} \sum_{\vec k} \cdots \rightarrow \int \frac{\diff^3k}{(2\pi)^3} \cdots \,.
\end{equation}

By finally comparing expression (\ref{eq:Gxt2Gpt}) with the definition of $ \Gt(\vec k, \tau) $ (\ref{eq:defGpt}), we identify

\begin{equation}
	\label{eq:GptExpression}
	\Gt(\vec k, \tau) = \theta(\tau) \langle \text{vac} | \hat c_{\vec k}(\tau) \hat c^\dagger_{\vec k}(0) | \text{vac} \rangle \,.
\end{equation}



\subsection{Properties}



According to the modified Heisenberg picture, .

The unity operator $ \hat \eye = \sum_\nu | \nu \rangle \langle \nu | $ may be expressed in terms of the complete set of eigenstates $ \{ | \nu (\vec k) \rangle \} $ to the grand canonical Hamiltonian $ \hat K \, | \nu (\vec p) \rangle = E_\nu(\vec p) \, | \nu (\vec p) \rangle $. Inserting this into expression (\ref{eq:GptExpression}), and recalling that the imaginary-time evolution of the annihilation operator, in the modified Heisenberg picture, is given by $ \hat c_{\vec k} (\tau) = e^{\hat K \tau} \, \hat c_{\vec k} (0) \, e^{- \hat K \tau} $ we obtain\cite{MishchenkoA.2000DqMC}

\begin{equation}
	\Gt(\vec k, \tau) = \theta(\tau)
	\sum_\nu | \langle \nu | \hat c^\dagger_{\vec k} | \text{vac} \rangle |^2 e^{-(E_\nu - E_0)\tau} \,.
\end{equation}

Here we have used that $ E_0 $ is the ground-state energy $ \hat K \, | \text{vac} \rangle = E_0 \, | \text{vac} \rangle $, and by shifting the energy scale it is permissible to set $ E_0 = 0 $.

The dependence on the imaginary-time difference of the single-particle Green's function is $ 2\beta $-periodic.This implies that there will be a corresponding infinite set of discrete frequencies $ \omega_n $ so that $ \Gt(\vec k, \tau) $ may be expressed as a Fourier series. However, letting $ T \rightarrow \infty $, the frequencies $ \omega_n $ may be treated as a continuous variable $ \omega $. Hence we might introduce the spectral function,
 
 \begin{equation}
	g_{\vec{k}} (\omega)
	\equiv \sum_\nu \delta (\omega - E_\nu) \, | \langle \nu | \hat c^\dagger_{\vec k} | \text{vac} \rangle |^2 \,.
\end{equation}

Assuming the single-particle Green's function to be that of an electron ($ \hat c_{\vec k} \rightarrow \hat a_{\vec k} $), the factor $ | \langle \nu | \hat a^\dagger_{\vec k} | \text{vac} \rangle $ is nothing more than the overlap of that particular eigenstate of $ \hat K $ onto a free electron with momentum $ \vec k $. In terms of therms of the spectral function, the Green's function is expressed as

\begin{equation}
	\Gt(\vec k, \tau) = \int_0^\infty g_{\vec k} (\omega) \, e^{-i \omega \tau} \diff \omega \,.
\end{equation}

From this it is clear that the spectral function must be proportional to $ \Gt(\vec k, \omega) = (E(\vec k) - i \omega)^{-1} $ in the $ V \rightarrow \infty $ limit, and thus has poles of frequencies corresponding to stable states.






\todo{fit to exponential  for large $ \tau $'s and introduce $ Z $-factor.8}

\todo{G and D derived in the appendix}

\section{Derivation of the Fröhlich Hamiltonian}


What will happen to a dielectric medium if one introduces a charged particle, and how will this particle react to changes in the dielectric medium? These are questions which Herbert Fröhlich answered in his 1954 paper \textit{Electrons in lattice fields}\cite{electronsInLatticeFields}. In this section we will outline the original derivation but also look into some details relevant to us.

\subsection{Classical picture}

The model of interest to this thesis is the following. Consider a diatomic crystal, e.g. rock salt. Such a crystal is made up out of ions which pairwise have a zero net charge. This is depicted in figure \ref{fig:illustrationOfDiatomicCrystal} below. Introducing a free electron into such a crystal, the electron will polarize the crystal due to it exerting an electric field. This sort of polarization may be described by a displacement vector for each lattice point. However, to simplify matters, it is preferred to treat this displacement vector as a continuous vector field.

\begin{figure}[H]
        \centering
        \begin{subfigure}[b]{0.45\textwidth}
                \includegraphics[width=\textwidth]{{"Images/Crystal illustration/description"}.pdf}
                \caption{The filled circles illustrate four types of charges. The color represent the charge sign; the ones to the left are positively charged and the ones to the right are negatively charged. The quantity of charge is represented by the size; the charges have the same charge quantity within each row, but the ones in the top row have a weaker charge than the ones in the bottom row.}
        \end{subfigure}\hfill
        \begin{subfigure}[b]{0.45\textwidth}
                \includegraphics[width=\textwidth]{{"Images/Crystal illustration/crystal"}.pdf}
                \caption{A layer of a diatomic crystal. This particular crystal has a face-centered cubic structure.}
        \end{subfigure}
        \caption{}
	\label{fig:illustrationOfDiatomicCrystal}
\end{figure}

If we first consider a steady state situation, the polarization is determined solely by the dielectric permittivity $ \varepsilon_r $. Introducing the electric displacement field $ \vec D = \vec E + 4 \pi \vec P $ (CGS units) for which the source is the free electron, this may be thought of as the external electrical field. If our electron is at position $ \vec r_\text{e} $ , the $ \vec D $-field at a position $ r $ is given by the usual

\begin{equation}
	\vec D( \vec r, \vec r_\text{e}) = - \nabla \frac{Q}{\left| \vec r - \vec r_\text{e} \right|}
\end{equation}

so that

\begin{equation}
	\nabla \cdot \vec D( \vec r, \vec r_\text{e}) = 4 \pi q \, \delta(\vec r - \vec r_\text{e}) \,.
\end{equation}

Here $ Q $ is the charge of the electron and the second equality is to be thought of as valid inside of an integral. In agreement with the definition of the $ \vec D $-field we may also define the electric field due to the bound charges as $ \vec E_\text{bound} = - 4 \pi \vec P $ since $ \vec D = \vec E + 4 \pi \vec P = \vec E - \vec E_\text{bound} \equiv \vec E_\text{free} $.

The interaction energy, which is minimised when the $ \vec D $-field is parallel to the polarisation, is given by \question{(why!? Dipole something?)}

\begin{equation}
	\label{eq:interaction_energy}
	E_\text{int} = - \int_V \vec D(\vec r, \vec r_\text{e}) \cdot \vec P(\vec r) \diff^3 r \,.
\end{equation}

By utilising the scalar potential of the bound electric field $ \vec E_\text{bound} (r) = - \nabla \Phi (\vec r) $ along with the quantities defined above, this interaction energy may be rewritten as

\begin{equation}
	\label{eq:interaction_energy_calculation}
	\resizebox{\hsize}{!}{$
	    	\begin{split}
    			E_\text{int} 
    			&= - \frac{1}{4 \pi} \int_\Omega \vec D(\vec r, \vec r_\text{e}) \cdot \nabla \Phi(\vec r) \diff^3 r \\
    			&= - \frac{1}{4 \pi} \int_{\Omega \setminus V_\varepsilon} \vec D(\vec r, \vec r_\text{e}) \cdot \nabla \Phi(\vec r) \diff^3 r
    				 - \frac{1}{4 \pi} \int_{V_\varepsilon} \vec D(\vec r, \vec r_\text{e}) \cdot \nabla \Phi(\vec r) \diff^3 r \\
    			&= - \frac{1}{4 \pi} \int_{\Omega \setminus V_\varepsilon} \nabla \cdot \left[ \Phi(\vec r) \vec D(\vec r, \vec r_\text{e}) \right] \diff^3 r
    				+ \underbrace{
    					\frac{1}{4 \pi} \int_{\Omega \setminus V_\varepsilon} \Phi(\vec r) \nabla \cdot \vec D(\vec r, \vec r_\text{e}) \diff^3 r
    				}_{0}
    				+ \underbrace{
    					\frac{q}{4 \pi} \int_{V_{\varepsilon}'} \frac{1}{r^2} \frac{\partial \Phi(\vec r + \vec r_\text{e})}{\partial r}  \diff^3 r
    				}_{\rightarrow \, 0 \; \text{as} \; \varepsilon \, \rightarrow \, 0} \\
    			&= \frac{Q}{4 \pi} \int_{-\partial V_{\varepsilon}'} \Phi(\vec r + \vec r_\text{e}) \nabla \frac{1}{r} \cdot \diff^2 \vec{r} \\
    			&= Q \, \Phi(\vec r_\text{e})
    		\end{split}
    	$}
\end{equation}

Here we have both assumed that $ \Phi (\vec r) $ has a continuous first derivative as well as $ \allowbreak \Phi(\vec r) \vec D(\vec r, \vec r_\text{e}) \simeq 0 $ on the boundary $ \vec r \in \partial \Omega $. In order to use the divergence theorem the singularity at $ \vec r_\text{e} $ has been isolated in a spherical volume $ V_\varepsilon $ centered at $ \vec r_\text{e} $ with radius  $ \varepsilon \rightarrow 0 $.

Next we bring back the time dependence and no longer consider a situation which is in steady state. Since each lattice point in our crystal is occupied by a ion, the dynamic in our system is characterized by two time scales. That is, the time it takes to displace the bound electrons relative their nuclei (deformation of ion) and the time it takes to displace the nucleus relative the lattice (deformation of lattice structure). These two types of deformation are illustrated in figure \ref{fig:twoTypesOfPolrisation} below. Since the electrons are significantly lighter than their nucleus, the ion deformation time should thus be much smaller than the time it takes to deformation the lattice. Denoting these times as $ t_\text{uv} $ and $ t_\text{ir} $ respectively we thus have that $ t_\text{uv} \ll t_\text{ir} $ which corresponds to $ \omega_\text{uv} \gg \omega_\text{ir} $. The subscripts of course indicate that frequencies lie in the ultraviolet and infrared region respectively. The total deformation is due to these two deformation so that $ \vec P = \vec P_\text{ir} + \vec P_\text{uv} $.

\begin{figure}[H]
        \centering
        \begin{subfigure}[b]{0.45\textwidth}
                \includegraphics[width=\textwidth]{{"Images/Crystal illustration/uv polarization"}.pdf}
                \caption{Deformation of the ions.}
        \end{subfigure}\hfill
        \begin{subfigure}[b]{0.45\textwidth}
                \includegraphics[width=\textwidth]{{"Images/Crystal illustration/ir polarization"}.pdf}
                \caption{Deformation of the lattice.}
        \end{subfigure}
        \caption{}
        \label{fig:twoTypesOfPolrisation}
\end{figure}

It is further reasonable to assume that each of these contributions behave as a driven harmonic oscillator, that is

\begin{equation}
	\label{eq:driven_harmonic_oscillator}
	\ddot{ \vec P}_\text{ir} (\vec r) + \omega^2_\text{ir} \, \vec P_\text{ir} (\vec r) = \frac{ \vec D(\vec r, \vec r_\text{e})}{ \gamma }
	\; , \quad
	\ddot{ \vec P}_\text{uv} (\vec r) + \omega^2_\text{uv} \, \vec P_\text{uv} (\vec r) = \frac{ \vec D(\vec r, \vec r_\text{e})}{ \delta } \,.
\end{equation}

Here $ \gamma $ and $ \delta $ are constants to be determined in what follows. First we must once again quickly consider a static situation. Here the static dielectric constant $ \epsilon $ would give us $ \vec D = \epsilon \vec E $ which we could use together with the previously stated definition of the $ \vec D $-field to relate $ \vec P $ and $ \vec D $ as

\begin{equation}
	\label{eq:static_P_D_relation}
	4 \pi \vec P(\vec r) = (1 - 1/\epsilon) \vec D(\vec r, \vec r_\text{e}) \,.
\end{equation}

Next we once again turn back time and utilise the high frequency dielectric constant $ \epsilon_\infty $, defined by $ \vec D = \epsilon_\infty \vec E $, under the assumption that the frequency $ \omega_\infty $ of the external field satisfies  $ \omega_\text{uv} \gg \omega_\infty \gg \omega_\text{ir} $. The latter is simply implying that the free electron in our crystal is moving slowly \motivation{(Fast enough for the IR polarisation not to follow, right!?)}. This tells us that the UV-part of the polarisation will follow the changes in the $ \vec D $-field nearly adiabatically whilst the IR-part wont have time adapt to the changes and may thus be thought of as a negligible constant field in comparison, that is, $ \vec P \simeq \vec P_\text{uv} $. Using this together with the relation of $ \vec D $ and $ \vec E $ through $ \epsilon_\infty $ we find, in similarity to (\ref{eq:static_P_D_relation}),

\begin{equation}
	\label{eq:static_Puv_D_relation}
	4 \pi \vec P_\text{uv} (\vec r) =  (1 - 1/\epsilon_\infty) \vec D(\vec r, \vec r_\text{e}) \,.
\end{equation}

Taking the difference between (\ref{eq:static_P_D_relation}) and (\ref{eq:static_Puv_D_relation}) we find a similar equation for $ \vec P_\text{ir} $

\begin{equation}
	\label{eq:static_Pir_D_relation}
	4 \pi \vec P_\text{ir} (\vec r) =  (1/\epsilon_\infty - 1/\epsilon) \vec D(\vec r, \vec r_\text{e}) \,.
\end{equation}

Noting that $ \ddot{\vec P}_\text{uv} \approx \omega_\infty^2 \, \vec P_\text{uv} \ll \omega_\text{uv}^2 \, \vec P_\text{uv} $ and assuming that $ \vec P_\text{ir} $ is seemingly constant at timescales of length $ 1/\omega_\text{ir} $, equation (\ref{eq:driven_harmonic_oscillator}) simplifies into \motivation{(in the original derivation they simply compared using steady state where the time derivatives obviously are zero, but this must be equivalent due to $ \omega_\text{uv} \gg \omega_\infty \gg \omega_\text{ir} $?)}

\begin{equation}
	\label{eq:driven_harmonic_oscillator}
	\omega^2_\text{ir} \, \vec P_\text{ir} (\vec r) = \frac{ \vec D(\vec r, \vec r_\text{e})}{ \gamma }
	\; , \quad
	\omega^2_\text{uv} \, \vec P_\text{uv} (\vec r) = \frac{ \vec D(\vec r, \vec r_\text{e})}{ \delta } \,.
\end{equation}

Comparing this equation to (\ref{eq:static_Puv_D_relation}) and (\ref{eq:static_Pir_D_relation}) we find the value of $ \gamma $ and $ \delta $ in terms of know material properties

\begin{equation}
	\frac{1}{\gamma} = \frac{\omega_\text{ir}^2}{4 \pi} \left( \frac{1}{\epsilon_\infty} - \frac{1}{\epsilon} \right)
	\; , \quad
	\frac{1}{\delta} = \frac{\omega_\text{uv}^2}{4 \pi} \left( 1 - \frac{1}{\epsilon_\infty} \right) \,.
\end{equation}

Having obtained the equation of motion for the two types of polarisation, our next task is to find the one for the free electron. Since the electron had low in kinetic energy, we may without hesitation treat it in the non relativistic limit. The only force acting on the electron is due to the bound electric field, that is $ \vec F(\vec r_\text{e}) = q \vec E_\text{bound} (\vec r_\text{e}) = - q \nabla \Phi (\vec r_\text{e}) $. Thus $ m \ddot{\vec r}_\text{e} = \dot{\vec p}_\text{e} = - q \nabla \Phi (\vec r_\text{e}) $ where $ m $ is an effective mass which in general is different from the electronic mass $ m_\text{e} $. By treating $ \vec p_\text{e} $,  $ \vec P_\text{uv} (\vec r) $ and $ \vec P_\text{ir} (\vec r) $ as variables, we may by combining all the equation of motions derived above together with expression for the interaction energy (\ref{eq:interaction_energy_calculation}) form the Lagrangian

\begin{equation}
	\begin{split}
		L &= \frac{\gamma}{2} \int \left[ \dot{\vec P}_\text{ir}^2 (\vec r) - \omega_\text{ir}^2 \, \vec P_\text{ir}^2 (\vec r) \right] \diff^3 r
		+ \frac{\delta}{2} \int \left[ \dot{\vec P}_\text{uv}^2 (\vec r) - \omega_\text{uv}^2 \, \vec P_\text{uv}^2 (\vec r) \right] \diff^3 r \\
		&+ \int \vec D(\vec r, \vec r_\text{e}) \cdot \left[ \vec P_\text{ir}(\vec r) +  \vec P_\text{uv}(\vec r) \right] \diff^3 r
		+ \frac{\vec p_\text{e}^2}{2m}
	\end{split}
\end{equation}

and by a Legendre transformation, the Hamiltonian

\begin{equation}
	\label{eq:Hamiltonian_raw}
	\begin{split}
		H &= \frac{\gamma}{2} \int \left[ \dot{\vec P}_\text{ir}^2 (\vec r) + \omega_\text{ir}^2 \, \vec P_\text{ir}^2 (\vec r) \right] \diff^3 r
		+ \frac{\delta}{2} \int \left[ \dot{\vec P}_\text{uv}^2 (\vec r) + \omega_\text{uv}^2 \, \vec P_\text{uv}^2 (\vec r) \right] \diff^3 r \\
		&- \int \vec D(\vec r, \vec r_\text{e}) \cdot \left[ \vec P_\text{ir}(\vec r) +  \vec P_\text{uv}(\vec r) \right] \diff^3 r
		+ \frac{\vec p_\text{e}^2}{2m}	\,.
	\end{split}
\end{equation}

Since the UV-part of the polarisation follows the external field nearly adiabatically, the dynamic solution $ \vec P_\text{uv} $ is approximatively the statical one with respect to current $ \vec D $-field. Because of this $ \vec P_\text{uv} $ will only contribute to the Hamiltonian with a constant and may thus be removed from the theory giving rise only to an energy shift.

Next we observe that the integrals in the expressions above are divergent since $ \vec P_\text{ir} $ according to (\ref{eq:static_Puv_D_relation}) and (\ref{eq:static_Pir_D_relation}) are proportional to the $ \vec D $-field which is singular at $ \vec r = \vec r_\text{e} $. However this divergence is a resulted from us disregarding the atomic structure when interpreting the displacements as a continuous vector field. To resolve this one should express the polarisation in a Fourier series and omit terms with a wave length smaller than that of the lattice constant. Hence we need not to worry about this divergence.

Further more it is useful to introduce here a complex vector field $ \vec B $,

\begin{equation}
	\label{eq:B_field_def}
	\vec B(\vec r) = \sqrt{\frac{\gamma \omega_\text{ir}}{2}} \left[ \vec P_\text{ir}(\vec r) + \frac{i}{\omega_\text{ir}} \dot{\vec P}_\text{ir}(\vec r) \right]
	\; , \quad
	\vec B^\dagger(\vec r) = \sqrt{\frac{\gamma \omega_\text{ir}}{2}} \left[ \vec P_\text{ir}(\vec r) - \frac{i}{\omega_\text{ir}} \dot{\vec P}_\text{ir}(\vec r) \right] \,.
\end{equation}

Here we have also dropped the subscript of $ \omega_\text{ir} $ since $ \omega_\text{uv} $ no longer appears in our theory. The reasons to why one prefer to work with $ \vec B $ rather than $ \vec P_\text{ir} $ will be evident later. With this our now non-divergent Hamiltonian becomes,

\begin{equation}
	H = \frac{1}{2} m \dot{\vec r}_\text{e}^2
	+ \omega \int \vec B^\dagger (\vec r) \cdot \vec B (\vec r) \diff^3 r
	- \sqrt{\frac{1}{2 \gamma \omega}} \int \vec D(\vec r, \vec r_\text{e}) \cdot \left[ \vec B(\vec r) +  \vec B^\dagger(\vec r) \right] \diff^3 r \,.
\end{equation}

Using a periodic boundary condition over a cube of volume $ V = l^3 $, $ \vec B $ might be expressed as

\begin{equation}
	\label{eq:B_field_fourier}
	\vec B (\vec r) = \sum_{\vec q} \frac{\vec q}{q} b_{\vec q} \, \frac{e^{i \vec q \cdot \vec r}}{\sqrt V}
	\; , \quad
	\vec B^\dagger (\vec r) = \sum_{\vec q} \frac{\vec q}{q} b^\dagger_{\vec q} \, \frac{e^{-i \vec q \cdot \vec r}}{\sqrt V}
\end{equation}

with

\begin{equation}
	\vec q_i = \frac{2\pi}{l} n_i
	\; ; \quad 
	n_i = 0, \, \pm 1, \, \pm 2, \cdots , \, \pm N_\text{cut}
\end{equation}

\todo{Refer to these mentioned earlier. However you need to introduce $ N_\text{cut} $.}

Here  $ i = x, y, z $, $ | \vec q | = q $ and the cutoff magnitude $ q_\text{cut} = 2 \pi N_\text{cut} / l $ is inversely proportional to the lattice constant in accordance to what was discussed earlier \motivation{($ a \lesssim \lambda = 2 \pi / k \rightarrow k \lesssim a $ right!?)} \todo{(to write: $ q_\text{cut} $ since we are only looking at the IR-part which is du to lattice deformations at length scales $ \propto  a $)}. Also defining a scalar potential for the IR-part of the polarization as $ 4 \pi \vec P_\text{ir} = \nabla \Phi_\text{ir} $ we may, by using equation (\ref{eq:interaction_energy_calculation}), express the interaction energy between $ \vec D $ and $ \vec P_\text{ir} $ as $ Q \, \Phi_\text{ir} (\vec r_\text{e}) $. Using (\ref{eq:B_field_def}, \ref{eq:B_field_fourier}) together with $ \Phi_\text{ir} $ the expression for the Hamiltonian becomes

\begin{equation}
	H = \frac{\vec p_\text{e}^2}{2m}
	+ \omega \sum_{\vec q} b^\dagger_{\vec q} b_{\vec q}
	+ \sum_{\vec q} V(q) \left[ b^\dagger_{\vec q} \, e^{-i \vec q \cdot \vec r_\text{e}} - b_{\vec q} \, e^{i \vec q \cdot \vec r_\text{e}} \right] \,,
\end{equation}

where we have substituted using

\begin{equation}
	V(q) = i \left( 2 \sqrt 2 \pi \alpha \right)^{1/2} \sqrt{ \frac{1}{V} } \left( \frac{\omega^3}{m} \right)^{1/4} \frac{1}{q}
\end{equation}

and introduced the interaction parameter

\begin{equation}
	\alpha = \frac{1}{2} \left( \frac{1}{\epsilon_\infty} - \frac{1}{\epsilon} \right) \sqrt{ \frac{2m}{\omega} } Q^2 \,.
\end{equation}

Here it seems as if the interaction part of the Hamiltonian is ill behaved due to the $ \vec q = \vec 0 $ term of the sum. This is a result from the periodic boundary condition imposed on our system. Originally the system contained only one free electron. With a periodic boundary condition however, the system now consists of an infinite number of copies of the cube with volume $ V $, each containing a free electron. These electrons interact with one another due to the long range Coulomb repulsion. To avoid this behaviour when periodically continuing the system, one should introduce in each cube a charge density of $ - Q/V $. Doing this would give rise to an attractive Coulomb force which for each electron would cancel the Coulomb repulsion. Such an interaction would exactly cancel the ill behaved $ \vec q = \vec 0 $ term in the interaction part of the Hamiltonian since the only difference would be the sign of the charge, i.e $ Q \rightarrow -Q $.

\subsection{Quantisation}

Having sorted out these details we are ready to quantise the Hamiltonian. To do so the first step is to symmetrise $ b^\dagger_{\vec q} b_{\vec q} = (b^\dagger_{\vec q} b_{\vec q} + b_{\vec q} b^\dagger_{\vec q})/2 $ \question{(why? source needed.)}. Next up, using\cite{superfluidStatesOfMatter}

\begin{equation}
	\dot q = \frac{\delta H}{\delta p}
	\; , \quad
	\dot p = - \frac{\delta H}{\delta q}
\end{equation}

together with (\ref{eq:driven_harmonic_oscillator}, \ref{eq:Hamiltonian_raw}) it is easy to verify that $ \vec P_\text{ir} $ and $ \gamma \dot{ \vec P}_\text{ir} $ or equivalent, $ (b^\dagger_{\vec q} + b_{\vec q})/\sqrt{2 \gamma \omega} $ and $ i (b^\dagger_{\vec q} - b_{\vec q}) \sqrt{\gamma \omega / 2} $ for each $ \vec q $, are conjugate variables \question{(how do we arrive at the final set of conjugate variables? Using some sort of transformation perhaps? Try find in the book by Egor.)}.  By promoting to operators $ b_{\vec q} \rightarrow \hat b_{\vec q} $, $ \vec r_\text{e} \rightarrow \hat{\vec r}_\text{e} $, $ \vec p_\text{e} \rightarrow \hat{\vec p}_\text{e} $ and imposing Bose statistics \question{(How do we know this? spinless?)} onto the conjugate variables,

\begin{equation}
	i \delta_{\vec q, \vec k}
	= \frac{i}{2} [\hat b^\dagger_{\vec q} + \hat b_{\vec q}, \, \hat b^\dagger_{\vec k} - \hat b_{\vec k}]/2
\end{equation}

we obtain $ [b_{\vec q}, \, b^\dagger_{\vec k}] = \delta_{\vec q, \vec k} $ and $ [b_{\vec q}, \, b_{\vec k}] = 0 $  \question{(really, they follow?)}. Using the first of these to rewrite $ (\hat b^\dagger_{\vec q} \hat b_{\vec q} + \hat b_{\vec q} \hat b^\dagger_{\vec q})/2  = \hat b^\dagger_{\vec q} \hat b_{\vec q} + \tfrac{1}{2} $ the Hamiltonian, in quantised form, becomes

\begin{equation}
	\hat H = \frac{\hat{\vec p}_\text{e}^2}{2m}
	+ \omega \sum_{\vec q} \hat b^\dagger_{\vec q} \hat b_{\vec q}
	+ \sum_{\vec q} V(q) \left( \hat b^\dagger_{\vec q} \, e^{-i \vec q \cdot \hat{\vec r}_\text{e}} - \hat b_{\vec q} \, e^{i \vec q \cdot \hat{\vec r}_\text{e}} \right) \,.
\end{equation}

Here we have omitted the zero point energy $ \omega \sum_{\vec q} \tfrac{1}{2} $ of the oscillating ions. At this stage it has become evident that $ \hat b^\dagger_{\vec q} $ and $ \hat b_{\vec q} $ may be recognised\cite{electronsInLatticeFields} as the creation and annihilation operator of a harmonic oscillator of mass $ M $, with position and momentum coordinates

\begin{equation}
	\hat q_{\vec k} = \sqrt{\frac{1}{2M\omega}} \left( \hat b^\dagger_{\vec k} + \hat b_{\vec k} \right)
	\; , \quad
	\hat p_{\vec k} = i\, \sqrt{\frac{M \omega}{2}} \left( \hat b^\dagger_{\vec k} - \hat b_{\vec k} \right) \,.
\end{equation}

The polarization field $ \vec P_\text{ir} $ is in this manner being represented by a set of harmonic oscillators each corresponding to a quantized mode of vibration in the lattice, i.e. a phonon. Since the frequency of the phonons is momentum independent they are said to be dispersionless \question{(right!?)}. For optical phonons which couple to infrared radiation, this is at low momenta a good approximation as shown in the figure \ref{fig:phononDispersionRelation} below.

\begin{figure}[H]
	\centering
 	\includegraphics[width=0.4\textwidth, ]{{"Images/Dispersion relation/dispersion relation"}.pdf}
	\caption{The upper and lower curves depict the characteristics of an optical and acoustic dispersion relation respectively. The outlined orange region illustrates were the optical dispersion relation might be approximated as momentum independent.}
	\label{fig:phononDispersionRelation}
\end{figure}

Having sorted out the quantisation of the $ \vec B $-field we turn our attention towards the electron. In order to later expand the interacting theory using Feynman diagrams, we which to represent the free electron by a many-body Fock state instead of having it represented by a single-body Hilbert state. In order to do this the second quantisation formalism need to utilised. Using the momentum basis as the electronic many-body Fock occupation state with corresponding creation and annihilation operators $ \hat a^\dagger_{\vec k} $ and $ \hat a_{\vec k} $ respectively, the translation of a one-body operator from first-quantisation to second-quantisation is\cite{quantumTheoryOfManyParticleSystems}

\begin{equation}
	\hat O_2 = \sum_{\vec k, \vec p} \langle \vec k | O_1 | \vec p \rangle \hat a^\dagger_{\vec k} \hat a_{\vec p} \,.
\end{equation}

Here $ \hat O_2 $ is the second-quantized operator and $ O_1 $ is the corresponding first-quantized operator sandwiched between the single-particle Hilbert states $ | \vec k \rangle $, $ | \vec p \rangle $. \todo{Mention that these momenta are discretized in the same way as before but without a cutoff}. Thus

\begin{equation}
	\frac{\hat{\vec p}_\text{e}^2}{2m} \rightarrow \sum_{\vec p} \frac{p^2}{2m} \hat a^\dagger_{\vec p} \hat a_{\vec p}
	\; , \quad
	e^{-i \vec q \cdot \hat{ \vec r}_\text{e}} \rightarrow \sum_{\vec p} \hat a^\dagger_{\vec p - \vec q} \hat a_{\vec p}
	\; , \quad
	e^{i \vec q \cdot \hat{ \vec r}_\text{e}} \rightarrow \sum_{\vec p} \hat a^\dagger_{\vec p + \vec q} \hat a_{\vec p}
\end{equation}

\motivation{
To be included in the appendix perhaps?

\begin{equation}
	\begin{split}
		e^{-i \vec q \cdot \hat{ \vec r}_\text{e}}
		& \rightarrow \sum_{\vec k, \vec p} \langle \vec k | e^{-i \vec q \cdot \hat{ \vec r}_\text{e}} | \vec p \rangle \hat a^\dagger_{\vec k} \hat a_{\vec p} \\
		& = \sum_{\vec k, \vec p} \langle \vec k |
			\left[ \int | \vec x \rangle \langle \vec x | \diff^3 x \right]
			e^{-i \vec q \cdot \hat{ \vec r}_\text{e}}
			\left[ \int | \vec y \rangle \langle \vec y | \diff^3 y \right]
			| \vec p \rangle \hat a^\dagger_{\vec k} \hat a_{\vec p} \\
		& = \sum_{\vec k, \vec p} \hat a^\dagger_{\vec k} \hat a_{\vec p}
			\int \diff^3 x \;
			\frac{e^{- i \vec k \cdot \vec x}}{\sqrt{V}} \,
			e^{-i \vec q \cdot \vec x} \,
			 \frac{e^{i \vec p \cdot \vec x}}{\sqrt{V}} \\
		& = \sum_{\vec k, \vec p} \hat a^\dagger_{\vec k} \hat a_{\vec p}
			\; \delta( \vec k + \vec q - \vec p) \\
		& = \sum_{\vec p} \hat a^\dagger_{\vec p - \vec q} \hat a_{\vec p}
	\end{split}
\end{equation}
}

We substitute this into the Hamiltonian, and make in the interaction term use of the fact that the summation over $ \vec q $ is symmetric, to finally obtain

\begin{equation}
	\hat H
	= \sum_{\vec p} \frac{p^2}{2m} \hat a^\dagger_{\vec p} \hat a_{\vec p}
	+ \omega \sum_{\vec q} \hat b^\dagger_{\vec q} \hat b_{\vec q}
	+  \sum_{\vec q, \vec p} V (q) \left( \hat b^\dagger_{\vec q} - \hat b_{- \vec q} \right) \hat a^\dagger_{\vec p - \vec q} \hat a_{\vec p} \,.
\end{equation}

\question{Where do we assume that the crystal is homogenous/isotropic? Do we even have to do that approximation?}

\todo{Take the limit $ l \rightarrow \infty $. Perhaps do this where we apply the algorithm? That is, the Feynman diagrams section?}
\question{When doing this, what happens to the $ \sqrt V $ in $ V(q) $? According to the other papers $ \sqrt{V} \rightarrow 1 $. However in the $ L \rightarrow \infty $ limit, we normalize the momentum states by the Dirac delta

\begin{equation}
	\frac{e^{i \vec q \cdot \vec r}}{\sqrt V} \rightarrow \frac{e^{i \vec q \cdot \vec r}}{(2 \pi)^{3/2}} \,.
\end{equation}

In that case, $ \sqrt V \rightarrow (2 \pi)^{3/2} $ and the Hamiltonians do not agree. Solve this by letting $ b_{\vec q} \rightarrow \sqrt{V} b_{\vec q} $. What is the meaning of $ \sqrt{V} $ in the Fourier expansion of $ \vec B $ anyway?
}
\motivation{Might be the case that we are using the convention

\begin{equation}
	\sum_{\vec q} \rightarrow \int \frac{\diff^3 q}{(2 \pi)^{3/2}}
\end{equation}

instead of the usual

\begin{equation}
	\sum_{\vec q} \rightarrow \int \frac{\diff^3 q}{(2 \pi)^3} \,.
\end{equation}
}
\question{Might also have something to do with the $ b_{\vec q} $'s being conjugate variables? But it would be weird, if you could not multiply both with a constant and still keep them being conjugate...}




\section{Feynman diagrams}


In the $ T = 0 $ formalism

\begin{equation}
	\begin{split}
		\frac{\langle \Psi_0 |  \hat O_\text{H}(t)  | \Psi_0 \rangle}{\langle \Psi_0  | \Psi_0 \rangle}
		&=
		\sum_{\nu = 0}^\infty \left( - i \right)^\nu \frac{1}{\nu !}
		\int_{-\infty}^{\infty} \diff t_1 \cdots \int_{-\infty}^{\infty} \diff t_\nu
		\, e^{- \epsilon \left( |t_1| + \cdots + |t_\nu| \right)} \\[1em]
		&\times \frac{\langle \Phi_0  | T[\hat H_\text{I} (t_1) \cdots \hat H_\text{I} (t_\nu) \, \hat O_\text{I}(t)] | \Phi_0 \rangle}{\langle \Phi_0  | \hat S |\Phi_0 \rangle}
	\end{split}
\end{equation}

Take limit $ \epsilon \rightarrow 0 $ if permissible (for us that is the case). Disconnected diagram cancel.

Looking at

\begin{equation}
	\langle \Psi_0 |  T[\hat a_{\vec k}(t) \hat a^\dagger_{\vec k}(0)]  | \Psi_0 \rangle
	\propto
	\langle \Phi_0  | T[\hat H_\text{I} (t_1) \cdots \hat H_\text{I} (t_\nu) \, \hat a_{\vec k}(t) \hat a^\dagger_{\vec k}(0)] | \Phi_0 \rangle
\end{equation}

where the momentum basis is used, the Feynman Diagrams follow from all nonzero possible sandwiches. Better even, use Wicks theorem and make contractions.




\section{Dyson equation}



\section{Monte Carlo Simulation}

\subsection{Detailed balance and ergodicity}

\subsection{Metropolis-Hastings algorithm}



\chapter{Diagrammatic Monte Carlo}

\todo{Here we shall summarise what this chapter is about}

\section{General idea}

\section{Generalisation of stochastic integration (better name needed)}

\subsection{Example 1}
\subsection{Example 2}
\subsection{Example 3}

\section{Polaron implementation}

\todo{Here we will demonstrate what happens when we are using $ T = 0 $ and not having any free electrons in the system at the beginning.}

\subsection{Update function 1}
\subsection{Update function 2}
\subsection{Update function 3}

\section{Sample self energy}

\subsection{Divergent diagram}

The first order proper self energy diagram is proportional to $ 1/\sqrt{\tau} $ for small $ \tau $ and thus diverges when $ \tau \rightarrow 0 $. This combined with a discretised histogram becomes a problem for the bins in the neighbourhood of $ \tau = 0 $ since they acquire more mass then they are supposed to. In order to fix this we could use a much finer discretisation or calculate the value for these bins somehow else.

The value of the diagram for $ \vec p = \vec 0 $ and $ \Delta G = 0 $ is

\begin{equation}
	S^{(1)}(\tau) = \frac{\alpha}{\sqrt{\pi \tau}} e^{(\mu - \omega)\tau}
\end{equation}

The value of the bins in our histogram is

\begin{equation}
	\tilde V_i \propto \frac{1}{\Delta \tau} \int_{\tau_i}^{\tau_{i + 1}} S^{(1)} (t) \diff t
	= \frac{\alpha}{\Delta \tau \sqrt{\omega - \mu}} \left[ \text{erf}\left(\sqrt{\omega - \mu} \sqrt{\tau_{i+1}}\right) - \text{erf}\left(\sqrt{\omega - \mu} \sqrt{\tau_i}\right) \right]
\end{equation}

 but the true value should be
 
\begin{equation}
	V_i \propto S^{(1)} \left( \tfrac{\tau_{i+1} + \tau_i}{2} \right)	
\end{equation}

For all bins which fulfil $ \tilde V_i  - V_i > \epsilon $ we instead of using DMC utilise ordinary stochastic integration using MC for the fixed times $ (\tau_{i+1} + \tau_i)/2 $.

\section{Boldification}




\chapter{Results}
\todo{Here we shall summarise what this chapter is about}

\section{Analytical predictions}

\section{Numerics}

\subsection{Discrete Fourier transform}

\todo{Make sure to increase vector lengths. Illustrate what might happen otherwise.}

\section{Discussion}


\chapter{Summary and Conclusions}
\todo{Here we shall summarise what this chapter is about}


\begin{appendices}
\chapter{(make this go away, also Appendix A -> Appendix)}

\section{Heavy theory from the finite temperature formalism}

\section{Derivation of the finite temperature electron Green's function}

\section{Derivation of the finite temperature phonon Green's function}
\end{appendices}

\bibliography{thebib}
\bibliographystyle{vancouver}

\end{document}